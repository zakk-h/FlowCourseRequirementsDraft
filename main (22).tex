\documentclass{beamer}

% Theme and Packages
\usepackage{graphicx}
%\usetheme{Boadilla}
%\usetheme{Antibes}
%\usetheme{Berlin}
%\usetheme{Copenhagen}
\usetheme{Malmoe}


\title{\textbf{Graph Modeling: Network Flows to Inform Course Selection}}
\author{\textbf{Zakk Heile and Isaac Yang} \ (Duke University)}
\date{\textbf{December 11, 2024}}

\begin{document}

% Title Slide
\begin{frame}
    \titlepage
\end{frame}

% Slide: Overview of Problems
\begin{frame}{Overview of Problems}
    \begin{itemize}
        \item Check if a selection of classes satisfies distribution requirements.
        \item Find the \textbf{minimum number} of classes to satisfy requirements.
        \item Optimize class selection to minimize time commitment or maximize enjoyment while satisfying requirements.
        \item Given an existing selection of classes, determine optimal ways to satisfy constraints.
    \end{itemize}
\end{frame}

\begin{frame}{Modeling Course Selection}
    \centering
    \LARGE We model the Course Selection problem as a Max Flow problem.
\end{frame}

% Slide for initialexample1.png with citation
\begin{frame}
    \centering
    \includegraphics[width=\textwidth]{initialexample1.png}
    \vspace{0.5cm}
    \tiny{Cited from Michael Sambol}
\end{frame}

% Slide for initialexample2.png
\begin{frame}
    \centering
    \includegraphics[width=\textwidth]{initialexample2.png}
\end{frame}

% Slide for initialexample3.png
\begin{frame}
    \centering
    \includegraphics[width=\textwidth]{initialexample3.png}
\end{frame}

% Slide for initialexample4.png
\begin{frame}
    \centering
    \includegraphics[width=\textwidth]{initialexample4.png}
\end{frame}

% Slide for initialexample5.png
\begin{frame}
    \centering
    \includegraphics[width=\textwidth]{initialexample5.png}
\end{frame}

% Slide for initialexample6.png
\begin{frame}
    \centering
    \includegraphics[width=\textwidth]{initialexample6.png}
\end{frame}

% Slide for initialexample7.png
\begin{frame}
    \centering
    \includegraphics[width=\textwidth]{initialexample7.png}
    \tiny (This is not a max flow, more steps to do but we leave it here). Time complexity: $O(f \cdot E)$ or $O(V \cdot E)$
\end{frame}

% Slide: Architecture
\begin{frame}{Architecture}
    \begin{itemize}
        \item \textbf{Source Node}: Connects to class nodes (infinite capacity).
        \item \textbf{Class Nodes}: Link to specific general requirement nodes (specific to the course):
        \begin{itemize}
            \item Modes of Inquiry
            \item Areas of Knowledge
            \item Seminar Requirements
        \end{itemize}
        \item \textbf{Requirement Nodes}: Flow general requirement nodes to shared nodes (e.g., SS, CCI).
        \item \textbf{Sink Node}: Collects all flow to confirm requirement satisfaction.
    \end{itemize}
\end{frame}

\begin{frame}
    \centering
    \includegraphics[width=\textwidth]{architecture.png}
\end{frame}


% Slide: Constraints
\begin{frame}{Constraints}
    \begin{itemize}
        \item Classes may fulfill multiple Modes of Inquiry (up to 3) or Areas of Knowledge (1).
        \item Given that courses may be coded with more than that amount, many allocations are possible for given classes, complicating flow conservation.
        \item \textbf{Foreign Languages}: Special rules:
        \begin{itemize}
            \item 3 courses at 100/200 level, or
            \item 1 course at 300+ level, which satisfies all 3 requirements.
        \end{itemize}
        \item Complex conservation rules:
        \begin{itemize}
            \item A 300+ level foreign language course fulfills the Foreign Language requirement with weight 3 without penalizing other Modes of Inquiry or Areas of Knowledge.
        \end{itemize}
    \end{itemize}
\end{frame}

\begin{frame}
    \centering
    \includegraphics[width=\textwidth]{FLproblem.png}
\end{frame}

% Slide: Binning
\begin{frame}{Binning}
    \begin{itemize}
        \item Many classes share the same codes, increasing the complexity of graph construction and max flow computation.
        \item Solution: Bin classes into sets of codes, focusing on the sets needed to satisfy requirements.
        \item A set of codes may be used multiple times (up to the number of available classes).
        \item Additional upper bounds for code set usage:
        \begin{itemize}
            \item \( 2 \times \text{numNonFLcodes} + 3 \times \text{ifFLCode} \)
            \item 12 (I can construct 12 courses to graduate)
        \end{itemize}
        \item How many copies? Minimum of all 3 of those bounds.
    \end{itemize}
\end{frame}

% Slide: Handling Pre-Taken Classes
\begin{frame}{Handling Already-Taken Classes}
    \begin{itemize}
        \item Codes used from already-taken classes are not fixed but can be assigned in multiple ways.
        \item Assign codes in all possible ways for the given classes:
        \begin{itemize}
            \item Possibilities for a class: \( \binom{4}{3} \) or \( \binom{2}{1} \), or rarely their product.
        \end{itemize}
        \item For each assignment:
        \begin{itemize}
            \item Build a graph, push flow through the assigned codes.
            \item Remove the used flow and reduce capacity accordingly.
            \item Decrease the remaining max flow required for graduation.
        \end{itemize}
        \item Evaluate all graphs and max flows and select the best.
    \end{itemize}
\end{frame}

% Slide for phase1.png
\begin{frame}{Handling Already-Taken Classes Efficiently}
    \centering
    \includegraphics[width=\textwidth]{phase1.png}
    \vspace{0.5cm}
    \small Idea: restrict valid \( s \)-to-\( t \) paths.
\end{frame}

% Slide for phase2.png
\begin{frame}
    \centering
    \includegraphics[width=\textwidth]{phase2.png}
    \vspace{0.5cm}
    \small This path is allowed, but backtracking to the course is not.
\end{frame}

% Slide: Enumerating All Max Flows
\begin{frame}{Enumerating All Max Flows}
    \begin{itemize}
        \item During max flow computation, at each iteration:
        \begin{itemize}
            \item Identify all possible \( s \)-to-\( t \) augmenting paths.
            \item For each path, branch into a new state by pushing flow along that path.
        \end{itemize}
        \item Recursively explore all branches, maintaining flow conservation and capacity constraints.
        \item Keep only the unique max flow distributions.
        \item Note: smartly choosing \( s \)-to-\( t \) paths can improve efficiency. For example, when there is still a path to get more codes on a class already locked in, don't explore new classes. 
    \end{itemize}
\end{frame}

\begin{frame}{Finding All Maximum Flows Elegantly}
    \begin{itemize}
        \item Given two maximum flows \( f_1 \) and \( f_2 \):
        \begin{itemize}
            \item \( f' = f_2 - f_1 \) is a circulation (\( \sum_{u \in V} f'_{uv} - \sum_{u \in V} f'_{vu} = 0 \)).
        \end{itemize}
        \item Thus, any two max flows differ by a circulation!
        \item Thus, if we have a max flow and can find all circulations, we can get every other max flow.
        \item Non-trivial circulations correspond to cycles in the residual graph:
    \end{itemize}
\end{frame}

\begin{frame}{Finding Max Flows with Minimum Classes}
    \begin{itemize}
        \item Goal: Identify maximum flows that use the smallest number of classes.
        \item If we compute all the max flows then we just need to filter out the ones using more than the minimum number of classes.
        \item This approach is computational, but there is no easy way to find all max flows subject to a constraint without it (that we can come up with).
        \item We want all max flows (specifically with bins) because there are not that many (if our upper bound on bin copies is tight) and the user may have a preference (I would!).
    \end{itemize}
\end{frame}

%TO DO - MAX FLOW SUBJECT  to minimizing number of classes, more detail on foreign language, actually code something to show

\begin{frame}{Revisiting Foreign Language Gadgets}
    \begin{itemize}
        \item \textbf{Key Insight:} If we compute all max flows, we can handle the Foreign Language (FL) constraints flexibly.
        \item \textbf{Strategy for 300 Level \(\leq\) 2nd class:}
        \begin{itemize}
            \item Include both FL gadgets:
            \begin{itemize}
                \item Capacity 3 with no FL option.
                \item Capacity 2 MOI no FL with Capacity 3 FL.
            \end{itemize}
        \end{itemize}
        \item \textbf{Process:}
        \begin{enumerate}
            \item Compute all possible max flows.
            \item Filter out flows that:
            \begin{itemize}
                \item Send partial flow to both gadgets.
            \end{itemize}
            \item Minimize objectives or return among the valid flows.
        \end{enumerate}
    \end{itemize}
\end{frame}


\begin{frame}
    \centering
    \includegraphics[width=\textwidth]{newfl.png}
    \tiny{We want to allow backtracking and switching between gadgets, unlike taken courses.}
\end{frame}

% Slide: Conclusion
\begin{frame}{Conclusion}
    \begin{itemize}
        \item Network flow models provide a powerful framework to optimize course selection even with obtuse constraints.

        \item Future work will focus on refining optimization techniques and algorithm efficiency. 
    \end{itemize}
    \vspace{1cm}
    \textbf{Contact Information:}
    \begin{itemize}
        \item Zakk Heile, Duke '27: \texttt{zakk.heile@duke.edu}
        \item Isaac Yang, Duke '26: \texttt{isaac.y@dukekunshan.edu.cn}
    \end{itemize}
\end{frame}

\end{document}